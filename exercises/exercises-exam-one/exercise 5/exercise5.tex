\documentclass[12pt]{article} 
\input{../custom}
\graphicspath{{figures/}}
\def\showcommentary{1}
\newcommand{\class}[1]{\mathscr{#1}}
\newcommand{\reals}{\mathbb{R}}
\newcommand{\ints}{\mathbb{Z}}
\newcommand{\tht}{\theta}
\newcommand{\tho}{\theta_0}
\newcommand{\thn}{\hat{\theta}_n}
\newcommand{\ps}{\mathbb{P}}


\title{Practice Exercise 5}
\author{}
\date{}


\begin{document}
\maketitle



\subsection*{Exercise}



Let 
\begin{align*}
p(y|\theta) &= \tht^{-1}e^{-y/\tht},\; y>0,\; \tht>0.\\
p(\tht) &= \tht^{-a}e^{-b/\tht},\; \tht>0,\;a>2,\;b>0.
\end{align*}

\begin{enumerate}
\item Find the posterior distribution of $\tht|y.$ 
\item Calculate the posterior mean and posterior variance. 
\item Notice the prior is still proper when $1<a\leq2.$ How would such a change affect the posterior mean and posterior variance?\\
\end{enumerate}

\newpage

\emph{Solution:}
\begin{enumerate}
\item $\theta|y \sim \text{InverseGamma}(a,y+b).$
\item $E(\theta|y) = \dfrac{y+b}{a-1}.$
$V(\theta|y) = \dfrac{(y+b)^2}{(a-1)^2(a-2)}.$
\item Suppose $1<a\leq2.$ Then the posterior mean still exists, but the posterior variance doesn't (since we need $a>2$).


\end{enumerate}


\end{document}



