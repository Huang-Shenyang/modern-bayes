\documentclass[12pt]{article} 
\input{../custom}
\graphicspath{{figures/}}
\def\showcommentary{1}


\title{Exercise}
\author{}
\date{}


\begin{document}
\maketitle

%\subsection*{Instructions}
%\begin{itemize}
%\item \textbf{Don't look at the solution yet!} This is for your benefit.
%\item This exercise must be submitted within 48 hours of the lecture in which it was given. 
%\item As long as you do the exercise on time, you get full credit---your performance does not matter.
%\item Without looking at the solution, take 5 minutes to try to solve the exercise.
%\item Pre-assessment: Write down how correct you think your answer is, from 0 to 100\%.
%\item Post-assessment: Now, study the solution and give yourself a ``grade'' from 0 to 100\%.
%\item Submit your work on the course website, including the pre- and post- assessments.
%\end{itemize}

\subsection*{Exercise}
You need to model the probability of recovery for patients admitted to the hospital in severe cardiac distress. (Let's say that ``recovery'' means the patient survived long enough to be discharged from the hospital.) You have past data from a number of patients from 6 different hospitals.
For each patient $i$, you have various information $x_i = (x_{i 1},\ldots,x_{i p})$ (e.g., gender, weight, age, blood pressure, etc.), and a binary outcome $y_i$ (did the patient recover or not).
Design a Bayesian model for this problem.


\newpage
\vfill
\rotatebox{180}{
\begin{minipage}[t][\textheight][t]{\textwidth}
\subsection*{Solution}\scriptsize
There are multiple ways of approaching this, and each will be better or worse depending on the particulars of the situation.
One natural way is to use a hierarchical Probit regression model, as follows. Let's divide up the patients according to hospital,
and let $y_{hi}$ be the outcome for the $i$th patient in hospital $h$, and let $x_{hi} = (x_{hi1},\ldots,x_{hip})$ be the corresponding information.
\begin{align*}
    & Y_{hi}|\theta_h \,\sim\, \Bernoulli(\Phi(\theta_h^\T x_{hi})) \text{ independently} \\
    & \theta_h|\mu,\Sigma \,\iid\, \N(\mu,\Sigma) \\
    & \mu \sim \N(\mu_0,\Sigma_0) \\
    & \Sigma \sim \mathrm{InverseWishart}(S,\nu).
\end{align*}
This would be appropriate if the effect of the predictors/covariates $x$ was expected to be different for the different hospitals.
~\\~\\
Another way would be to augment the covariate vector with indicator variables for the hospital, i.e., for $j = 1,\ldots,6$, define
$x_{h,i,p+j} = \I(j = h)$, and use a Probit regression model with a Normal prior on $\theta = (\theta_1,\ldots,\theta_{p+6})$.
This would be appropriate if the effect of the predictors/covariates was expected to be the same for the different hospitals,
but the overall recovery rate varied among hospitals.
\end{minipage}}

\end{document}






