\documentclass[12pt]{article} 
\input{../custom}
\graphicspath{{figures/}}
\def\showcommentary{1}


\title{Exercise}
\author{}
\date{}


\begin{document}
\maketitle

%\subsection*{Instructions}
%\begin{itemize}
%\item \textbf{Don't look at the solution yet!} This is for your benefit.
%\item This exercise must be submitted within 48 hours of the lecture in which it was given. 
%\item As long as you do the exercise on time, you get full credit---your performance does not matter.
%\item Without looking at the solution, take 5 minutes to try to solve the exercise.
%\item Pre-assessment: Write down how correct you think your answer is, from 0 to 100\%.
%\item Post-assessment: Now, study the solution and give yourself a ``grade'' from 0 to 100\%.
%\item Submit your work on the course website, including the pre- and post- assessments.
%\end{itemize}

\subsection*{Exercise}
Consider the following Exponential model for an observation $x$:
$$ p(x|a,b) = a b \exp(- a b x) \I(x>0)$$
and suppose the prior is 
$$ p(a,b) = \exp(- a - b)\I(a,b>0). $$
You want to sample from the posterior $p(a,b|x)$.  Find the conditional distributions needed for implementing a Gibbs sampler.


\newpage
\vfill
\rotatebox{180}{
\begin{minipage}[t][\textheight][t]{\textwidth}
\subsection*{Solution}\scriptsize
The Gibbs sampler consists of alternately sampling from $a|b,x$ and $b|a,x$. First note that the joint p.d.f.\ is
$$ p(x,a,b) = a b \exp(-a b x - a - b) \I(a,b,x>0). $$
Thus,
\begin{align*}
    p(a|b,x) &\underset{a}{\propto} p(x,a,b) \underset{a}{\propto} a \exp(-a b x - a)\I(a>0) = a \exp(-(b x + 1)a)\I(a>0)
    \underset{a}{\propto} \Ga(a\mid 2,\, b x + 1).
\end{align*}
Therefore, $p(a|b,x) = \Ga(a\mid 2,\,b x+1)$ and by symmetry, $p(b|a,x) = \Ga(b\mid 2,\,a x+1)$.
\end{minipage}}

\end{document}






